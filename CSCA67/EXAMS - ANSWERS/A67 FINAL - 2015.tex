\documentclass{article}
\usepackage{amsfonts}
\usepackage{tikz}
\usepackage{amsmath}
\usepackage{amssymb}
\usepackage{setspace}
\usepackage{mathtools}
\usepackage{fancyhdr}
\usepackage{fullpage}
\usepackage{graphicx}
\usepackage{caption}
\usepackage{lipsum}
\usepackage{venndiagram}
\usepackage{booktabs}
\usepackage[usestackEOL]{stackengine}[2013-10-15]
\def\x{\hspace{3ex}}    %BETWEEN TWO 1-DIGIT NUMBERS
\def\y{\hspace{2.45ex}}  %BETWEEN 1 AND 2 DIGIT NUMBERS
\def\z{\hspace{1.9ex}}    %BETWEEN TWO 2-DIGIT NUMBERS
\graphicspath{ {./images/} }
\stackMath
\def\checkmark{\tikz\fill[scale=0.4](0,.35) -- (.25,0) -- (1,.7) -- (.25,.15) -- cycle;}

\begin{document}

\begin{titlepage}
\begin{center}
  \vspace{5cm}

  Discrete Mathematics, Fall 2015

  \hrulefill
  \vspace{0.3cm}

  \textbf{\huge CSCA67 - FINAL EXAM \& ANSWERS}

  \hrulefill

  \vspace{0.5cm}

  \includegraphics[scale = 0.25]{UTSC}

  \textbf{GOOD LUCK EVERYONE!}

  -Richard

  \vspace{4cm}

  \end{center}

  \textbf{Instructors:}

  \begin{center}

  \begin{tabular}{lll}
   &\textbf{Dr. Anna Bretscher}&\textbf{Dr. Richard Pancer}\\
  \textbf{Email:} & bretscher@utsc.utoronto.ca & pancer@utsc.utoronto.ca\\
  \textbf{Office:} & IC493 & IC490\\
  \textbf{Office Hours:}& Monday 12:10 - 1:30 & Monday 11:10 - 12:30\\
   & Wednesday 1:10 - 2:00 & Friday 1:30 - 3:00\\
   & Friday 1:10 - 2:00 (will change after week 6)&
  \end{tabular}

  \end{center}

\end{titlepage}
\pagestyle{fancy}

\vspace{0.5cm}
\chead{Richard Hong}
\lhead{CSCA67 - EXAM 2015}
\rhead{\leftmark}
\cfoot{}
\rfoot{Page \thepage}

\text{}

\begin{enumerate}


\item Your local grocery store just received a large shipment of apples, oranges, pears, and bananas—more than
100 pieces each. You are shopping at the store and will purchase your fruit for the week.

\begin{enumerate}
  \item How many ways can you select 10 pieces of fruit from your store’s supply of apples, oranges, pears,
and bananas?

let $f_i$ denote a kind of fruit

$f_1 + f_2 + f_3 + f_4 = 10, f_i \geq 0, i \in [1,4]$

Total = $\displaystyle{{10 + (4 - 1)}\choose{4 - 1}}$

\item How many ways can you select 10 pieces of fruit from your store’s supply of apples, oranges, pears,
and bananas if you need at least one piece of each kind of fruit?

let $f_i$ denote a kind of fruit

$\displaystyle f_1 + f_2 + f_3 + f_4 = 10, f_i \geq 1, i \in [1,4] \equiv f_1 + f_2 + f_3 + f_4 = 6, f_i \geq 0, i \in [1,4]$

Therefore, Total = $\displaystyle{{6 + (4 - 1)}\choose{4 - 1}}$

\end{enumerate}

\item In how many ways can six 3’s and four 2’s be arranged in a row so that the 2’s are always apart?

Let's write this as $\square 3 \square 3 \square 3 \square 3 \square 3 \square 3 \square 3 \square$

We see that we have 7 $\square$s to fill our four 2s in, so that the 2s are always apart.

$\therefore$ ways to arrange six 3’s and four 2’s be arranged in a row so that the 2’s are always apart is $\displaystyle{{7}\choose{4}}$



\item Your Toronto Maple Leafs won 30 of 82 games last season (i.e., the 2014-2015 season), giving them a
winning percentage of 37\%. If we assume this means the probability of the Leafs winning any given game
is 0.37, then we can predict how they would have done in a playoff series.


Answer the following questions to determine the probability that the Leafs would have won a best of 7
playoff series (i.e., won 4 games) had they made the playoffs last season.

\begin{enumerate}

\item Rephrase this question in terms of sequences of 0s and 1s.

What is the shortest length of a sequence? 4. (4 wins)

What is the longest length of a sequence? 7. (7 combination of wins and loses)


\item Calculate the number of sequences which correspond to the Leafs winning the series. (Note that the
answer is not C(7, 4).)



\item Calculate the number of sequences as they relate to this problem. (Note that the answer is not 2
7 as not all series would last 7 games.)

\item Calculate the probability that the Leafs would win the series.

\item What is your best guess for the probability that the Leafs will ever win the Stanley Cup again (the
ultimate prize in the NHL)?

\end{enumerate}

\pagebreak

\text{}

\item For each of the following statements, list the numbered region(s) of the venn diagram that represent(s) when
the statement is true.

\begin{enumerate}

\vspace{1cm}

\begin{minipage}{0.5\linewidth}
\item $P \land Q \land \neg R$

\begin{venndiagram3sets}[tikzoptions={scale=0.6}, labelA={P}, labelB={Q}, labelC={R}]
\fillACapBNotC
\end{venndiagram3sets}

\vspace{0.5cm}

\end{minipage}
\begin{minipage}{0.5\linewidth}
\item $P \rightarrow Q \rightarrow R$

Assuming this is $(P \rightarrow Q) \rightarrow R$.

$(P \rightarrow Q) \rightarrow R \equiv \neg(\neg P \lor Q) \lor R \equiv (P \land \neg Q) \lor R$

\begin{venndiagram3sets}[tikzoptions={scale=0.6}, labelA={P}, labelB={Q}, labelC={R}]
\fillANotB
\fillC
\end{venndiagram3sets}
\end{minipage}

\vspace{0.5cm}


  \item $\neg((P \land Q) \rightarrow \neg R)$

$\neg((P \land Q) \rightarrow \neg R) \equiv \neg(\neg(P \land Q) \lor \neg R) \equiv \neg(\neg P \lor \neg Q \lor \neg R) \equiv P \land Q \land R$

\begin{venndiagram3sets}[tikzoptions={scale=0.6}, labelA={P}, labelB={Q}, labelC={R}]
\fillACapBCapC
\end{venndiagram3sets}

\end{enumerate}

\item Logical equivalence and contrapositive.

\begin{enumerate}

\item Circle the statements below that are equivalent to $(a \land b) \rightarrow c$.

1. $(\neg a \lor \neg b) \rightarrow \neg c$

\textbf{2.} $\neg c \rightarrow (\neg a \lor \neg b)$ \checkmark. $(a \land b) \rightarrow c \equiv \neg (\neg a \lor \neg b) \rightarrow c \equiv \neg c \rightarrow (\neg a \lor \neg b)$

\textbf{3.} c is neccesary for a and b \checkmark. $a \rightarrow b \equiv$ b is neccesary for a

4. c is sufficient for a and b

\textbf{5.} a and b are sufficient for c \checkmark. $a \rightarrow b \equiv$ a is sufficient for b

6. a and be are neccesary for c

7. all of the above

8. none of the above

\item If the implication $a \rightarrow b$ is difficult to prove directly, we can attempt a proof by contrapositive. Use a
truth table to show that the contrapositive of $a \rightarrow b$ is equivalent to $a \rightarrow b$.

Show $a \rightarrow b \equiv \neg b \rightarrow \neg a$

\begin{center}


\begin{tabular}{l|l|l|l}
$a$ & $b$ & $a \rightarrow b$ & $\neg b \rightarrow \neg a$\\
\hline
T & T & T & T\\
T & F & F & F\\
F & T & T & T\\
F & F & T & T\\
\end{tabular}
\hspace{0.5cm}
(c)
\begin{tabular}{l|l}
$a \rightarrow b$ & Given\\
$\neg a \lor b$ & Conditional or $(\rightarrow)$ Law\\
$b \lor \neg a$ & Commutative Property\\
$\neg b \rightarrow \neg a$ & Conditional or $(\rightarrow)$ Law\\
\end{tabular}



\end{center}

$\therefore a \rightarrow b \equiv \neg b \rightarrow \neg a $

\end{enumerate}

\pagebreak

\text{}

\item Modular arithmetic.

\begin{enumerate}

  \item State the Division Theorem. (Hint: This is the theorem we often use when writing proofs involving
the mod operator.)

$\forall n \in \mathbb N, n = kp + r, k \in \mathbb N, p \in \mathbb N, r \in [0,n)$

\item Prove that $\forall a, b, c \in \mathbb N, a \equiv_n b \rightarrow ca \equiv_n cb$


\begin{tabular}{l}
Assume $a \equiv_n b$\\
By the Division Theorem:\\
$a = k_1 \cdot n + r, b = k_2 \cdot n + r, k_1, k_2 \in \mathbb Z$\\
$c \cdot a = c \cdot k_1 \cdot n + c \cdot r$\\
$c \cdot b = c \cdot k_2 \cdot n + c \cdot r$\\
$c \cdot a \text{ mod } n = c \cdot r$\\
$c \cdot b \text{ mod } n = c \cdot r$\\
$\therefore ca \equiv_n cb$
\end{tabular}

\end{enumerate}

\item Consider the following statement S: The negative of any irrational number is irrational.

\begin{enumerate}
  \item Write S as an implication

$\forall n \in \mathbb R, n \in \mathbb I \rightarrow -n \in \mathbb I$

  \item


\end{enumerate}

\end{enumerate}

\end{document}
