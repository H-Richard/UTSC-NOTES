\documentclass{article}
\usepackage{amsfonts}
\usepackage{tikz}
\usepackage{amsmath}
\usepackage{amssymb}
\usepackage{setspace}
\usepackage{mathtools}
\usepackage{fancyhdr}
\usepackage{fullpage}
\usepackage{graphicx}
\usepackage{caption}
\usepackage{lipsum}
\usepackage{venndiagram}
\usepackage{booktabs}
\usepackage[usestackEOL]{stackengine}[2013-10-15]
\def\x{\hspace{3ex}}    %BETWEEN TWO 1-DIGIT NUMBERS
\def\y{\hspace{2.45ex}}  %BETWEEN 1 AND 2 DIGIT NUMBERS
\def\z{\hspace{1.9ex}}    %BETWEEN TWO 2-DIGIT NUMBERS
\graphicspath{ {./images/} }
\stackMath

\begin{document}

\begin{titlepage}
\begin{center}
  \vspace{5cm}

  \Large {\textbf{Course Notes}}\\
  \hrulefill
  \vspace{0.3cm}

  \textbf{\huge CSCA67 - Discrete Mathematics}

  \hrulefill

  \vspace{0.5cm}

  \includegraphics[scale = 0.25]{UTSC}

  \vspace{9cm}

  \end{center}

  \textbf{Instructors:}

  \begin{center}

  \begin{tabular}{lll}
   &\textbf{Dr. Anna Bretscher}&\textbf{Dr. Richard Pancer}\\
  \textbf{Email:} & bretscher@utsc.utoronto.ca & pancer@utsc.utoronto.ca\\
  \textbf{Office:} & IC493 & IC490\\
  \textbf{Office Hours:}& Monday 12:10 - 1:30 & Monday 11:10 - 12:30\\
   & Wednesday 1:10 - 2:00 & Friday 1:30 - 3:00\\
   & Friday 1:10 - 2:00 (will change after week 6)&
  \end{tabular}

  \end{center}

\end{titlepage}
\pagestyle{fancy}

\vspace{0.5cm}
\chead{Richard Hong}
\lhead{CSCA67 - Course Notes}
\rhead{\leftmark}
\cfoot{}
\rfoot{Page \thepage}

\text{}


\section{Propositions, Implications}

\textbf{Definitions}:

A \textbf{proposition} is a statement that evaluates to True or False. In computer science, its often referred to as a \textbf{Boolean expression}.

A \textbf{compound roposition} is a proposition statementt that involves multiple propositions joined by connectives. It takes multiple truth values as input and returns a single truth value as output.

A \textbf{connective} corresponds to English conjunctions such as "and", "or", "not" etc.

\noindent

\vspace{0.5cm}

\textbf{\large Basic connectives and truth tables:}

\begin{center}
\begin{tabular}{c|l}
Symbol & Meaning\\
\hline
$\land$ & "AND"\\

$\lor$ & "OR"\\

$\rightarrow$ & "IF...THEN"\\

$\leftrightarrow$ & "IF AND ONLY IF"\\

$\neg$ & "NOT"\\

\end{tabular}
\vline
\begin{tabular}{c|c|c|c|c|c}
$P$ & $Q$ & $P \land Q$ & $P \lor Q$ & $P \rightarrow Q$ & $P \leftrightarrow Q$\\
\hline
T & T & T & T & T & T\\
T & F & F & T & F & F\\
F & T & F & T & T & F\\
F & F & F & F & T & T\\
\end{tabular}
\end{center}

\vspace{0.3cm}

\hrulefill

\textbf{\large Implication:}

\begin{center}
{\textbf{Different ways of writing $P \rightarrow Q$:}}

\vspace{0.3cm}

\begin{minipage}[c]{0.4\textwidth}
\vspace{0pt}
\begin{tabular}{ll}
1. If P then Q & 6. Q is neccesary for P\\
2. If P, Q & 7. If not Q, then not P \\
3. Q, if P & 8. Not P or Q \\
4. P only if Q & \\
5. P is sufficient for Q &
\end{tabular}
\vspace{0pt}
\end{minipage}
\hspace{2cm}
\begin{minipage}[c]{0.3\textwidth}
\vspace{0pt}
\begin{venndiagram2sets}[labelA={P},labelB={Q}]
\fillNotB
\end{venndiagram2sets}
\vspace{0pt}
\end{minipage}

\end{center}

\hrulefill

\textbf{\large Logical Equivalences:}

\vspace{0.2cm}

\begin{tabular}{l|c|c}
Commutative  & $p\land q \iff q \land p$                  & $p \lor q \iff q\lor p$ \\
Associative  & $(p\land q)\land r \iff p \land (q\land r)$ & $(p \lor q)\lor r \iff p \lor (q \lor r)$\\
Distributive & $p \land (q \lor r) \iff (p \land q) \lor (p \land r)$ & $p \lor (q \land r) \iff (p \lor q) \land (p \lor r)$ \\
Identity     & $p \land T \iff p$                                 & $p \lor F \iff p$\\
Negation     & $p \lor \neg p \iff T$  & $p \land \neg p \iff F$\\
Double Negative & $\neg (\neg p) \iff p$ \\
Idempotent   & $p \land p \iff p$ & $p \lor p \iff p$\\
Universal Bound & $p \lor T \iff T$ & $p \land F \iff F$\\
De Morgan's  & $\neg(p \land q) \iff (\neg p)\lor (\neg q)$ & $\neg( p \lor q ) \iff (\neg p) \land (\neg q)$\\
Absorption   & $p \lor (p \land q) \iff p$ & $p \land (p \lor q)\iff p$\\
Conditional  or & $(p \to q) \iff (\neg p \lor q)$ & $\neg (p \to q) \iff ( p \land \neg q)$\\
($\to$) Law &&\\
Biconditional & $(p \leftrightarrow q) \iff (p \to q) \land (q\to p)$ \\
\end{tabular}

\vspace{0.4cm}

\hrulefill

\textbf{\large Order of Operations:}

\vspace{0.2cm}

1. NOT($\neg$) \hspace{1cm} 2. AND($\land$) \hspace{1cm} 3. OR($\lor$) \hspace{1cm} 4. Quantifiers($\forall/\exists$) \hspace{1cm} 5. ($\rightarrow/\leftrightarrow$)

\pagebreak

\text{}

\section{Predicates and Quantifiers}

\begin{center}
\begin{tabular}{ll}
\textbf{Forall:}& $\forall$\\
\textbf{There exists:}& $\exists$\\
\hline
Negations:\\
$\neg\forall = \exists$ & $\neg\exists = \forall$
\end{tabular}
\vline
\begin{tabular}{l}
\textbf{Prove statement in the form of} $\exists x \in S, \ni: P(x)$\\
We simply need to find \textbf{one} value of $x$ in the set $S$, that makes $P(x)$ true.\\
\textbf{One value is enough.}\\
\hline
\textbf{Example}:\\
There exists an integer $n$, such that $n^2$ is even.\\
$\exists n \in \mathbb Z, \ni: n^2 \in 2 \mathbb Z$\\
Let $n = 2$, then $(2)^2 = 4$ which is an even number
\end{tabular}

\vspace{0.5cm}

\begin{tabular}{l}
\textbf{Prove statemnet in the form of} $\forall x \in S, \ni: P(x)$\\
This means we must use techniques such as algebraic manipulation to show that:\\
$P(x)$ holds for every arbitrary $x \in S$\\
\hline
\textbf{Example}:\\
Forall integers $n$, if $n$ is even, then $n^2$ is even.\\
$\forall n \in \mathbb Z, n = 2k, k \in \mathbb Z \rightarrow n^2 = 2z, z \in \mathbb Z$\\
Let $n = 2k, k \in \mathbb Z$\\
then $n^2 = (2k)^2 = 4k^2$ which is an even number.\\
Therefore: Forall integers $n$, if $n$ is even, then $n^2$ is even. $QED$
\end{tabular}

\end{center}

\hrulefill

\subsection{Modulus}

\centerline{10 \textbf{mod} 3 = 1}
\noindent

The modulus or "mod" operator means the remainder when we divide two numbers.

\textbf{Congruent mod} means that two numbers have the same remainder when divided by one number.

\centerline{10 $\equiv_3$ 7 $\Leftrightarrow$ 10 mod 3 = 7 mod 3}

\subsection{Fundamental Theorem of Arithmetic}

\begin{minipage}{0.65\textwidth}
The \textbf{Fundamental Theorem of Arithmetic} states that any integer greater than 1 is either a \textbf{prime} number itself, or can be represented as the unique product of prime numbers.

For example:

\begin{center}

\begin{tabular}{cl}
16 & $= 2^4$\\
18 & $= 2^1 \cdot 3^2$\\
21 & $= 2^0 \cdot 3^1 \cdot 5^0 \cdot 7^1$
\end{tabular}
\end{center}

Numbers that can be written as the unique product of primes are called \textbf{Composite Numbers}.

\scriptsize{Reminder: a \textbf{prime number} is an number that can only be divided evenly by 1 and the number itself. }
\end{minipage}
\hspace{0.3cm}
\begin{minipage}{0.35\textwidth}
\includegraphics[scale = 0.6]{euclid}
\end{minipage}

\pagebreak

\text{}

\section{Basic Proof Strategies}


\begin{tabular}{l}
{\Large To prove in the form of $P \rightarrow Q$}:\\

 \\

\textbf{Direct Proof}: Assume $P$ is true then prove $Q$\\
\hline
This form works because if we recall the truth table for $P \rightarrow Q$,\\
When $P$ is true, $Q$ must be true for the statement to evaluate to true.\\

 \\

\textbf{Proof by Contrapositve}: Assume $\neg Q$ is true then prove $\neg P$\\
\hline
This form works because the contrapositive is logically equivalent to the original,\\
$P \rightarrow Q \equiv \neg Q \rightarrow \neg P$\\

 \\

\textbf{Proof by Contradiction}: Assume $\neg(\neg(P \rightarrow Q)) \equiv P \land \neg Q$\\
\hline
Then we must derive some sort of contradiction.\\
Once we arrive at the contradiction, that means one of our assumptions cannot be correct.\\
for example if $\neg Q$ is false, that means $Q$ is true.\\

 \\

\textbf{Proof by Cases/Exhaustion}: $X \lor Y \rightarrow Q$ Show $X \rightarrow Q \land Y \rightarrow Q$\\
\textbf{Example:}\\
$x \in \mathbb Z \rightarrow x^2 + x + 1\in 2 \mathbb Z + 1$
($x^2 + x$ is odd)\\
\hline
\textbf{Case 1:} x is odd\\
$x = 2k + 1$\\
$(2k + 1)^2 + (2k + 1) + 1$\\
$= 4k^2 + 6k + 3$\\
$= 2(2k^2 + 3) + 3$ case holds when x is odd. \\
\hline
\textbf{Case 2:} x is even\\
$x = 2k$\\
$(2k^2) + 2k + 1$\\
$= 4k^2 + 2k + 1$\\
$= 2(2k^2 + k) + 1$ case holds when x is even.\\
Since we have proven both case are indepdently even, we can conclude $\forall x \in \mathbb Z, x^2 + x + 1 \in 2\mathbb Z +
1$
\end{tabular}

\hrulefill

\hspace{1cm}

\textbf{ Some Definitions:}

\textbf{Theorem:} A statement that has already been proved.

\textbf{Axiom:} A statement that is self evidently true.

\textbf{Identiy:} An equation that is true for all values of an arbitrary variable.

\textbf{Proof:} A mathematical argument demonstrating the truth of a proposition.

\textbf{Tautology:} A propositional logic formula that always evaluates to True. ($A \lor \neg A$) - (I'm hungry or I'm not hungry)

\textbf{Rational Number:} A number that can be represented as the fraction of two relatively prime integers.

\centerline{$A \in \mathbb Q \rightarrow A = \frac{m}{n}, n\neq 0,  m, n \in \mathbb Z, gcd(m,n) = 1$}

\pagebreak

\text{}

\vspace{3cm}

\begin{center}

\textbf{Logic in a nutshell}

\includegraphics[scale = 0.22]{logic}

\begin{flushright}

{\scriptsize Graph from Introduction to mathematical arguments - by Michael Hutchings}

\end{flushright}

\end{center}

\pagebreak

\text{}

\section{Proof of Irrationality}

\subsection{Approach 1 - Fundamental Theorem of Arithmetic}

\begin{center}
\begin{tabular}{l}
Prove that $\sqrt{2}$ is irrational.\\
\hline

Assume the contrary that $\sqrt{2}$ is rational.\\
Then by the definition of rational numbers, $\sqrt{2} = \frac{m}{n}, \ni: m ,n \in \mathbb Z, gcd(m,n) = 1, n \neq 0$\\
\end{tabular}

\vspace{0.3cm}

{\large
$\sqrt{2} = \frac{m}{n}$\\
$n\sqrt{2} = m$\\
$2n^2 = m^2$\\}

\onehalfspacing

$m = x_1^{\alpha_1} \cdot x_2^{\alpha_2} \cdot x_3^{\alpha_3} \dots \cdot x_n^{\alpha_n}$ \hspace{0.3cm} $n = y_1{^\beta_1} \cdot y_2^{\beta_2} \cdot y_3^{\beta_3} \dots \cdot y_n^{\beta_n}$\\
Each x, y are primes by the fundamental theorem of arithmetic.\\
$m^2 = (x_1^{\alpha_1} \cdot x_2^{\alpha_2} \cdot x_3^{\alpha_3} \dots \cdot x_n^{\alpha_n})(x_1^{\alpha_1} \cdot x_2^{\alpha_2} \cdot x_3^{\alpha_3} \dots \cdot x_n^{\alpha_n})$\\
This means that $m^2$ has $2n$ possible factors.\\
$2n^2 = 2(y_1{^\beta_1} \cdot y_2^{\beta_2} \cdot y_3^{\beta_3} \dots \cdot y_n^{\beta_n})(y_1{^\beta_1} \cdot y_2^{\beta_2} \cdot y_3^{\beta_3} \dots \cdot y_n^{\beta_n})$\\
This means that $n^2$ has $2n$ possible factors plus one factor $2$.\\
as $m^2$ has an even number of prime factors, $2n^2$ will have an odd number of prime factors, contradicting the fundamental theorem.\\
$\therefore \sqrt{2} \in \mathbb I$ by contradiction. \marginpar{$QED$}


\vspace{0.5cm}

\end{center}

\subsection{Approach 2 - Definition of a Rational Number}

\begin{center}
\begin{tabular}{l}
Prove that $\sqrt{2}$ is irrational.\\
\hline

Assume the contrary that $\sqrt{2}$ is rational.\\
Then by the definition of rational numbers, $\sqrt{2} = \frac{m}{n}, \ni: m ,n \in \mathbb Z, gcd(m,n) = 1, n \neq 0$\\
\end{tabular}

\end{center}

$gcd(m,n)$ means that $m,n$ MUST be relative prime.

\begin{center}

{\large
$\sqrt{2} = \frac{m}{n}$\\
$n\sqrt{2} = m$\\
$2n^2 = m^2$\\
$2n^2 = m^2 \Rightarrow m^2 \in 2 \mathbb Z \Rightarrow m \cdot m \in 2 \mathbb Z$\\}
\hrulefill

{\normalsize The previousline showed that $m$ is even, so now we can substitute $m$ with any arbitrary even number $2k$.\\}
{\large
$m = (2k), k \in \mathbb Z$\\
$2n^2 = (2k^2)$\\
$2n^2 = 4k^2$\\
$n^2 = 2k^2$\\
$n^2 \in 2 \mathbb Z \Rightarrow n \in 2 \mathbb Z$\\
$m, n \in 2 \mathbb Z \Rightarrow gcd(m,n) \neq 1$\\
}
Since $m, n$ are both even, they cannot be relatively prime,
$\therefore \sqrt{2} \in \mathbb I$ by contradiction. \marginpar{$QED$}

\end{center}

\pagebreak

\text{}

\section{Induction}

\textbf{Simple Induction Format:}

Suppose we need to prove $P(n)$ for all natural numbers.

\begin{center}
\begin{tabular}{l}
\textbf{1. State the Predicates}\\
$P(n): \dots$\\

 \\

\textbf{2. Base case}\\
Prove that $P(n)$ holds when $n$ is the smallest possible natural number.\\
$P(0): \dots$ is True.\\

 \\

\textbf{3. Inductive Hypothesis}\\
Assume that $P(k)$ holds for any arbitrary $k$\\
$P(k): \dots$ is True.\\

 \\

\textbf{4. Inductive Step}\\
Prove that $P(k) \rightarrow P(k + 1)$\\
Assume $P(k)$ then show $P(k+1)$\\

\end{tabular}
\end{center}

\textbf{Example:} Prove   $\displaystyle \sum_{i = 0}^{n} i = \frac{n(n+1)}{2}$

\vspace{0.4cm}

\begin{center}
\begin{tabular}{l}
\textbf{Stating the Predicate: }$P(n): \displaystyle\sum_{i = 0}^{n} i = \frac{n(n+1)}{2}, \forall n \in \mathbb N$\\
 \\
\textbf{Base case: } $n = 0: \displaystyle\sum_{i = 0}^0 i = 0 \hspace{0.5cm} \frac{0(0 + 1)}{2} = 0$\\

 \\
\textbf{Inductive Hypothesis: } Assume for any arbitrary $k \geq 0$, $P(k)$ holds.\\

$P(k) = \displaystyle\sum_{i = 0}^{k} k = \frac{k(k+1)}{2}$

 \\

\textbf{Inductive Step: }Prove $P(k) \rightarrow P(k + 1)$\\

\\

$P(k + 1) = \displaystyle\sum_{i = 0}^{k + 1} i = 1 + 2+ 3 \dots + k + (k + 1)$\\

\\

$P(k + 1) = \frac{k(k+1)}{2} + (k + 1)$ \textbf{by Inductive Hypothesis}\\

\\

$P(k + 1) = \frac{k(k+1)+ 2(k+1)}{2}$\\

$P(k + 1) = \frac{(k+1)(k+2)}{2}$\\

\textbf{Conclusion:}\\
$\therefore P(k) \rightarrow P(k+1)$\\
$\displaystyle\sum_{i = 0}^{n} i = \frac{n(n+1)}{2}, \forall n \in \mathbb N$\\
\end{tabular}
\end{center}

\pagebreak

\text{}

\section{Pigeonhole Principle}

\onehalfspacing

\textbf{Core Principle}: There exists $n$ pigeons and $m$ pigeonholes, if $n > m$, there must be atleast one pigeonhole with atleast two pigeons.

\textbf{Example:} Prove that if 7 distinct numbers are selected from $\{1,2, \dots 11\}$, then some two will add to 12.

\textbf{Pigeons:} 7 \textbf{distinct numbers}

\textbf{Pigeonholes:} 6 \textbf{sets} of numbers that add up to 12.

\begin{center}
$\{1,11\},\{2,10\},\{3,9\},\{4,8\},\{5,7\},\{6\}$
\end{center}

\textbf{Note:} If we select 7 numbers from a set of 6, we will be forced to select atleast 2 of the numbers from the same set.

$\therefore$ if 7 distinct numbers are selected from $\{1,2, \dots 11\}$, then atleast two will add up to 12.

\marginpar{$QED$}

\vspace{2cm}



\pagebreak{}

\text{}

\section{Proof Samples}

\hrulefill

\subsection{Euclid's Proof for Infinite Primes}

\begin{center}
  \begin{tabular}{l}
  Assume to the contrary that there are a finite number of primes,\\
  then let this be the complete set of primes: $p_1, p_2, p_3 \dots p_n$\\
  \hline
  let $A = (p_1 \cdot p_2 \cdot p_3 \cdot p_4 \dots, \cdot p_n) + 1$\\
  $A$ is not divisible by any known primes as it always leaves a remainder of 1.\\
  so either $A$ is a prime number itself, or $A$ has a unique prime factor that is not in the existing list.\\
  \hline
  Contradictions:\\
  if $A$ is a prime number, then $p_n$ is not the greatest prime.\\
  if $A$ is a composite number, then $p_1, p_2, p_3 \dots p_n$ does not contain all the primes.\\
  Therefore, there must be an infinite number of primes.\\
  \end{tabular}
\end{center}

\marginpar{$QED$}

\subsection{Arithmetic mean and Geometric mean}

\hrulefill

\begin{center}
\begin{minipage}{0.35\textwidth}

\textbf{Defintion.} The arithmetic mean of $a_1$, $a_2$:

$$\frac{a_1 + a_2}{2}$$

\textbf{Definition.} The geometric mean of $a_1$, $a_2$:
$$\sqrt{a_1 \cdot a_2}$$
\end{minipage}
\hspace{0.3cm}
\vline
\begin{minipage}{0.5\textwidth}
  \hspace{0.3cm}
{\large \renewcommand{\arraystretch}{1.5}
\begin{tabular}{r@{$\geq$}l}
$\textbf{Prove that: }\forall a_1,a_2 \in \mathbb Z^+, \displaystyle\frac{a_1 + a_2}{2} $&$ \sqrt{a_1 \cdot a_2}$\\
$\displaystyle\frac{a_1 + a_2}{2} $&$ \sqrt{a_1 \cdot a_2}$\\
$(\displaystyle\frac{a_1 + a_2}{2})^2 $&$ a_1 \cdot a_2$\\
$\displaystyle\frac{ {a_1}^2 + 2(a_1 \cdot a_2) + {a_2}^2 }{4} $&$ a_1 \cdot a_2$\\
${a_1}^2 + 2(a_1 \cdot a_2) + {a_2}^2 $&$ 4 (a_1 \cdot a_2)$\\
$ {a_1}^2 + 2(a_1 \cdot a_2) + {a_2}^2 - 4 (a_1 \cdot a_2) $&$ 0$\\
${a_1}^2 - 2(a_1 \cdot a_2) + {a_2}^2 $&$ 0$\\
$(a_1 - a_2)^2 $&$ 0$\\
\end{tabular}
}
\end{minipage}

\end{center}

\marginpar{$QED$}

\pagebreak

\text{}

\section{Strong Induction}

\textbf{Strong Induction Format:}

Suppose we need to prove $P(n)$ for all natural numbers.

\begin{center}
\begin{tabular}{l}
\textbf{1. State the Predicates}\\
$P(n): \dots$\\

 \\

\textbf{2. Base cases}\\
Unlike Simple Induction, Strong Induction requires multiple base cases to be proven, \\
starting with the smallest value for $n$.\\
\textbf{Important note:} Dr.Pancer says natural numbers start at 1.\\

 \\

\textbf{3. Inductive Hypothesis}\\
Assume that $P(k)$ holds for any arbitrary $k$\\
$P(k): \dots$ is True.\\

 \\

\textbf{4. Inductive Step}\\
Prove that $P(k) \rightarrow P(k + 1)$\\
Assume $P(k)$ then show $P(k+1)$\\
A more formal way to write the Inductive Step for Strong Induction:\\
$P(1) \land P(2) \land P(3) \dots \land \dots P(k) \rightarrow P(k+1)$, where $P(1), P(2), P(3)$ were your base cases.\\
You may have to use your \textbf{Inductive Hypothesis} more than once.\\
\end{tabular}

\vspace{0.5cm}

Example on next page

\end{center}

\pagebreak

\text{}

\vspace{0.5cm}

\textbf{Example:} An alien race of hopkinsville goblins only have 4-cent and 5-cent coins in their currency. Prove that any amount of money greater than 12 cents, can be made using only 4-cent and 5-cent coins.

\begin{center}
\begin{tabular}{l}
\textbf{1. Stating the Predicates}\\
$P(n): \forall n \in \mathbb N, \ni: n \geq 12, n = 4a + 5b$\\

 \\

\textbf{2. Base cases}\\
$P(12) = 4(3) + 5(0)$\\
$P(13) = 4(2) + 5(1)$\\
$P(14) = 4(1) + 5(2)$\\
$P(15) = 4(0) + 5(3)$\\

 \\

\textbf{3. Inductive Hypothesis}\\
Suppose that $P(k)$, for any arbitrary $k$ where $k \in [15, n)$\\
$P(k): k = 4a + 5b$\\

 \\

\textbf{4. Inductive Step}\\
$P(k) \rightarrow P(k+4)$\\
$k = 4a + 5b$ Suppose $P(k)$ by \textbf{Inductive Hypothesis}\\
$k + 4 = 4a + 5b + 4$\\
$k + 4 = 4(a + 1) + 5b$\\
Since $k$ is arbitrary, $\forall n\in \mathbb N, P(n)$
\end{tabular}

\end{center}

Therefore, any amount of money greater than 12 cents, can be made using 4-cent and 5-cent coins by strong induction.

\marginpar{$QED$}

\pagebreak

\text{}

\section{The Sum Rule}

When counting, we need to be able to determine whether to sum or multiply the number of objects.

\textbf{The Sum Rule}. If an operation can be perforned in $n$ different ways, each having $x_1$ possible outcomes, then the total number of outcomes possible is:

\centerline{$\displaystyle\sum_{i = 1}^{n} x_i = x_1 + x_2 \dots + x_n$}

\vspace{0.2cm}

\textbf{Example.} Ordering pizza. Suppose a pizza shop offers 5 types of toppings and one has the choice of 3 topping, 2 topping or 1 topping.

Duplicate toppings are not allowed.

Order of toppings does not matter.

Determine how many different pizzas can be orderd.

\textbf{Note:} On the final, if a problem feels like it's unspecified, you are allowed to make your own assumptions, and answer them according to assumptions.

If you are right, you will not be penalized.

\begin{center}
\begin{tabular}{c|c}
Number of Toppings & Number of Pizza Choices ($x_i$)\\
\hline
&\\
1 & $5$\\
 & \\
2 & $\displaystyle\frac{5 \cdot 4}{2!}$\\
 & \\
3 & $\displaystyle\frac{5 \cdot 4 \cdot 3}{3!}$\\
\end{tabular}
\end{center}

\section{The Product Rule}

\textbf{The Product Rule}. Suppose an operation takes $k$ steps and that:

\quad - The first step can be performed $x_1$ ways.

\quad - The second step can be performed in $x_2$ ways.

Then the Whole operation can be performed in:

\centerline{$\displaystyle\prod_{i = 1}^{k} x_1 = x_1 \cdot x_2 \dots \cdot x_k$}

\vspace{0.2cm}

\textbf{Example.} Given 4 cities A, B, C and D. Suppose:

\quad - 5 routes between A and B

\quad - 3 routes between B and C and

\quad - 4 routes between C and D

How many different routes are there from A to D?

Number of AD routes = $5 \cdot 3 \cdot 4 = 60$

\pagebreak

\text{}

\textbf{Q.} We have seen two different counting scenarios - creating a Google password and ordering apair of pizzas. Which one involved the sum rule and which one involved the product rule?

\textbf{A.} The password problem involved the product rule, the pizza problem used both.

\section{Arrangement}

An arrangement is a grouping of objects. There are two types of arrangements:

\textbf{Definition.} A \textbf{Permutation} is an arrangement in which order matters.

\textbf{Definition.} A \textbf{Combination} is an arrangement in which order does not matter.

\section{Permutations}

\textbf{Definition.} An r-permutation of n distincts objects is an orderd arrangement of r of the n objects. We use the notation $P(n,r)$

The formula is derived as $n \cdot (n-1)_1 \cdot (n-2)_2 \dots \cdot (n-r + 1)_r$

$P(11,3) = 11 \cdot 10 \cdot 9$, $11 - 3 + 1$, $n = 11, r = 3$

\vspace{1cm}

\textbf{Q.} In terms of factorials, how can we rewrite this formula?

\textbf{A.} $P(n,r) = \displaystyle\frac{n!}{(n-r)!}$

\vspace{1cm}

\section{Combinations}

\textbf{Defintion.} An r-cominbation of n distinct objects is an unorderd selection, or a subset of r of the n objects.

We can think of combinations in terms of permutations.

\vspace{1cm}

\textbf{Q.} Given $P(n.r)$, the number of r-permutations of n objects, how can we derive the number of $C(n,r)$ of r-combinations of n objects?

\vspace{0.3cm}

\textbf{A.} $C(n,r) = \displaystyle\frac{P(n,r)}{r!} = \displaystyle\frac{n!}{(n-r)!\cdot sr!}$

\vspace{0.3cm}

$C(n,r) = {C_r}^n \equiv n(C_r) \equiv {n \choose r}$

\pagebreak

\text{}

\textbf{Example.} In Poker each player has 5 cards. A standard deck has 52 cards. How many 5-card hands are possible?

$C(52,5) = {52 \choose 5} = \displaystyle\frac{52!}{47! \cdot 5!} = 2,598,960$

\vspace{0.5cm}

\textbf{A flush} is when all 5 cards have the same suit. If there are 4 suits, i.e., 13 cards per suit, how many ways are there to obtain a flush?

\vspace{0.3cm}

$4 \cdot C(13,5) = 4 \cdot \displaystyle\frac{13!}{8! \cdot 5!}$ = 5148

Possiblity of a flush = $\displaystyle\frac{5148}{2,598,960} \approx 0.2\%$

\vspace{0.5cm}

\textbf{Q.} Should flushes happen very often?

\textbf{A.} No.

\vspace{0.5cm}

\textbf{Exercise.} How many different 8-digit binary sequences are there with six 1s and two 0s?

for example: $11011101$

8 bins must be filled in with six 1s and two 0s.

\vspace{0.3cm}

We should consider only filling in the 0s, as the 1s should fill in the rest of the space accordingly.

This is a combination:

\vspace{0.1cm}

$C(8,2) = C(8,6) = \displaystyle\frac{8!}{2! \cdot 6!}$

\hspace{8cm} Bin technique - Pancer

\section{Pascal's Triangle}


\begin{minipage}{0.55\linewidth}
\textbf{Blaise Pascal} [1623-1662] was a French mathematician, physicist, inventor, writer and philosopher

\vspace{0.2cm}

 - As a teenager, he invented the
mechanical calculator.

 - He collaborated with Pierre de
Fermat in Probability Theory
influencing modern economics
and social sciences.

 - Invented Pascal’s Triangle in his
“Treatise on the Arithmetic Triangle”.


\end{minipage}
\hspace{0.5cm}
\begin{minipage}{0.4\textwidth}
\Longstack[l]{
n=0\\
n=1\\
n=2\\
n=3\\
n=4\\
n=5\\
n=6\qquad\ \\
}
\Longstack{
1\\
1\x 1\\
1\x 2\x 1\\
1\x 3\x 3\x 1\\
1\x 4\x 6\x 4\x 1\\
1\x 5\y 10\z 10\y 5\x 1\\
1\x 6\y 15\z 20\z 15\y 6\x 1\\
\overline{0\x 1\x 2\x 3\x 4\x 5\x 6}
}
\end{minipage}

\vspace{0.3cm}

\textbf{Q.} What do you notice about each number? How is it related to
pizza and toppings?

\textbf{A.} $(n,r)^{th}$ element = C(n,r)

\pagebreak

\text{}

\textbf{Q.} How is Pascal’s Triangle related to binomial expansion? I.e., how is it related to the coefficients of the polynomials found by expanding $(a + b)^n$?

\textbf{A.}\begin{tabular}{l}
$(a + b)^1 = 1a + 1b$\\
$(a + b)^2 = 1a^2 + 2ab + 1b^2$\\
$(a + b)^3 = 1a^3 + 3a^2b + 3b^2a + 1b^3$\\
$\dots$
\end{tabular}

\vspace{0.3cm}

\textbf{Q.} What do each line sum to?

\textbf{A.} $1(2^0), 2(2^1), 4(2^2), 8(2^3) \dots$ powers of 2

\vspace{0.3cm}

\textbf{Q.} Read each line as a number (catenate). If a number has two digits carry the tens digit to the left and add. What do these numbers represent?

\textbf{A.} $1, 11, 121, 1331, 14541 \dots$ powers of 11.

\vspace{0.3cm}

\textbf{Q.} Colour all the odd numbers. What does this remind you of?

\textbf{A.} A well known repeating pattern, the Sierpinkski Triangle.

\vspace{0.3cm}

  \textbf{Q.} What series of numbers do you get by adding up the numbers
of the same colour which are on a stretched diagonal?

\textbf{A.} The Fibonacci sequence

Pascal was not the first to discover the triangle of binomial coefficients but was given credit because of how he related it to his work with probability and expectation.

\pagebreak

\text{}

\section{Probability}

The definition we will use was first defined by the French mathematician Pierre-Simon Laplace. He is famous for his work in astronomy, statistics and physics:

- Laplace tranform, Laplace’s equation

- First to postulate the existence of black holes

- inductive reasoning based on probability, today called Bayesian probability which plays a large role in artificial intelligence.

\textbf{Definition.} An experiment is a clearly defined procedure that results in one of a possible set of outcomes or elementary events.

\textbf{Definition.} A sample(probability) space of a random experiment is a set S that includes all possible outcomes of the experiment.

\textbf{Example.} If the experiment is to throw a standard die and record the outcome then:

sample space S = \{1,2,3,4,5,6\} elementary events

\textbf{Definition.} A compound event is a subset of S consisting of several elementary events.

\textbf{Q.} Using the experiment of throwing a die, what is an example of a compound event?

\textbf{A.}

Throwing an even number, E = \{2,4,6\}

Throwing a prime number, E = \{2,3,5\}

Throwing a power of 2, E = \{1,2,4\}

\vspace{0.3cm}


\textbf{Definition.} Let S be the sample space of an experiment and E be an event in S then Laplace’s definition of probability says that the probability of E is:

$$Prob(E) = \dfrac{|E|}{|S|}$$

\textbf{Q.} What is the probability of rolling a 3 on a standard die?

\textbf{A.}

$E = \{3\}, S = \{1,2,3,4,5,6\}$

$|E| = 1 \hspace{0.2cm} |S| = 6 \hspace{0.2cm} Prob(E) = \dfrac{1}{6}$

\vspace{0.3cm}


\textbf{Q.} What is the probability of rolling a power of two on a standard die?

\textbf{A.}

$E = \{1,2,4\}, S = \{1,2,3,4,5,6\}$

$|E| = 3 \hspace{0.2cm} |S| = 6 \hspace{0.2cm} Prob(E) = \dfrac{3}{6} = \dfrac{1}{2}$

\pagebreak

\text{}

\textbf{Bayes’ Rule.}

\textbf{Theorem (Bayes’ Rule).} Let A and B be events in the same
sample space. If neither P(A) nor P(B) are zero, then:

\centerline{$\dfrac{P(B) \cdot P(A|B)}{P(A)}$}

The second is the concept of total probability:

Theorem (Total Probability). Let a sample space S be a disjoint
union of events $E+1, E+2,...,E_n$ with positive probabilities, and let
$A \subset S$. Then:

$P(A) = \displaystyle \sum_{i = 1}^{n} P(A|E_i) \cdot P(E_i)$



\section{Complexity Analysis}

\onehalfspacing

When we talk about how good an algorithm is, we often use the
terms worst case complexity. This means, the number of steps
the algorithm takes for the worst possible input.

Sometimes we care more about the average or expected number
of steps.

Example. Searching a list L of integers from left to right. In the following list:

\centerline{L = 3, 5, 23, 6, 4, 1, 7, 10, 26, 8, 9, 11, 15}

if we search by visiting each integer from left to right we will perform
6 visits if we are looking for the number 1 and only 2 visits
if we are searching for 5.

\vspace{0.3cm}

\textbf{Q.} How many integers do we visit in the worst case?

\textbf{A.} In general, for a list of length n, n comparisons. (len(L)) Happens when the item we are looking for is on the far right of the list.

\vspace{0.3cm}

\textbf{Q.} If we assume that we are searching for an integer k in the list and that all positions of the list are equally likely to hold the integer we are looking for, what is the probability of finding the integer in the ith position of a list of length n?

\textbf{Expected Average} it's in the list and everything is equally likely.

\textbf{A.} $\dfrac{1}{n}$

\vspace{0.3cm}

\textbf{Q.} How many steps(comparisons) does it take to find the integer if it is in the
ith position?

\textbf{A.} Call this $N(i)$

We compute the expected E(n) number of steps of the algorithm
by computing for each of the i positions:

\centerline{P(k in the i
th position) · N(i)}

and then taking the sum over all n possible values for i

\begin{tabular}{l}
  E(n) = $\displaystyle \sum_{i = 1}^{n} P(\text{k in the ith position}) \cdot N(i)$\\
= $\displaystyle \sum_{i = 1}^{n} \dfrac{1}{n} \cdot i$\\
= $\displaystyle \dfrac{1}{n} \cdot \dfrac{n(n+1)}{2} = \dfrac{n(n + 1)}{2}$
\end{tabular}




\end{document}
