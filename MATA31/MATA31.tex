\documentclass{article}
\usepackage{amsfonts}
\usepackage{tikz}
\usepackage{amsmath}
\usepackage{amssymb}
\usepackage{setspace}
\usepackage{mathtools}
\usepackage{fancyhdr}
\usepackage{fullpage}
\usepackage{graphicx}
\usepackage{caption}
\usepackage{lipsum}
\usepackage{venndiagram}
\usepackage{booktabs}
\DeclareMathOperator{\sech}{sech}
\DeclareMathOperator{\csch}{csch}
\DeclareMathOperator{\arcsec}{arcsec}
\DeclareMathOperator{\arccot}{arcCot}
\DeclareMathOperator{\arccsc}{arcCsc}
\DeclareMathOperator{\arccosh}{arcCosh}
\DeclareMathOperator{\arcsinh}{arcsinh}
\DeclareMathOperator{\arctanh}{arctanh}
\DeclareMathOperator{\arcsech}{arcsech}
\DeclareMathOperator{\arccsch}{arcCsch}
\DeclareMathOperator{\arccoth}{arcCoth}
\graphicspath{ {./images/} }

\begin{document}

\begin{titlepage}
\begin{center}
  \vspace{5cm}

  {\textbf{MATA31}}\\

  Calculus 1, for Mathematical Sciences, Fall 2018

  \hrulefill
  \vspace{0.3cm}

  \textbf{\huge Various Core Proofs/Identities}

  \hrulefill

  \begin{tabular}{ll}
  Current Instructor: & Natalia Breuss\\
  \end{tabular}

  \vspace{0.5cm}

  \includegraphics[scale = 0.25]{UTSC}

  \vspace{5cm}

  \end{center}

  \textbf{Instructor(s):}

  \begin{center}

  \begin{tabular}{ll}
   &\textbf{Dr. Natalia Breuss}\\
  \textbf{Email:} & n.breuss@utoronto.ca\\
  \textbf{Office:} & IC484\\
  \textbf{Office Hours:}& Wednesday 11:00 - 13:00\\
  \end{tabular}

  \end{center}

\end{titlepage}
\pagestyle{fancy}

\vspace{0.5cm}
\chead{Richard Hong}
\lhead{MATA31 - Core Proofs/Identities}
\rhead{\leftmark}
\cfoot{}
\rfoot{Page \thepage}

\text{}

\section{Identities}
\begin{center}

\textbf{Basic Trigonometric Identiteis}

\begin{minipage}{0.3\linewidth}
Reciprocal Identities\\
\begin{tabular}{l}
$(\sin x)^{-1} = \csc x$\\
$(\cos x)^{-1} = \sec x$\\
$(\tan x)^{-1} = \cot x$
\end{tabular}
\end{minipage}
\begin{minipage}{0.3\linewidth}
Pythagorean Identities\\
\begin{tabular}{l}
$\sin^2 x + \cos^x = 1$\\
$\tan^2 x + 1 = \sec^2 x$\\
$1 + \cot^2 x = \csc^2 x$
\end{tabular}
\end{minipage}
\begin{minipage}{0.3\linewidth}
Angle Addition Identities\\
\begin{tabular}{l}
$\sin(x \pm y) = \sin x \cdot \cos y \pm \cos x \cdot \sin y$\\
$\cos(x + y) = \cos x \cdot \cos y - \sin x \cdot \sin y$\\
$\cos(x - y) = \cos x \cdot \cos y + \sin x \cdot \sin y$\\
$\tan(x + y) = \displaystyle\frac{\tan x + \tan y}{1 - \tan x \cdot \tan y}$\\
$\tan(x - y) = \displaystyle\frac{\tan x - \tan y}{1 + \tan x \cdot \tan y}$
\end{tabular}
\end{minipage}

\vspace{0.5cm}

\textbf{Hyperbolic Trigonometric Identities}

\vspace{0.3cm}

\begin{tabular}{ll}
$\sinh x = \dfrac{e^x - e^{-x}}{2}$ & $\csch x = \dfrac{2}{e^x - e^{-x}}$\\
 $\cosh x = \dfrac{e^x + e^{-x}}{2}$ & $\sech x = \dfrac{2}{e^x + e^{-x}}$\\
 $\tanh x = \dfrac{e^x - e^{-x}}{e^x + e^{-x}}$ & $\coth x = \dfrac{e^x + e^{-x}}{e^x - e^{-x}}$
\end{tabular}

\vspace{0.5cm}

\textbf{Derivatives of Trigonometric Functions}

\vspace{0.3cm}

\begin{minipage}{0.4\linewidth}
Derivative of Trigonometric Functions\\
\begin{tabular}{ll}
$(\sin x)' = \cos x$ & $(\csc x)' = -\csc x \cot x$\\
$(\cos x)' = -\sin x$ & $(\sec x)' = \sec x \tan x$\\
$(\tan x)' = \sec^2 x$ & $(\cot x)' = -\csc^2 x$
\end{tabular}
\end{minipage}
\hspace{0.3cm}
\begin{minipage}{0.55\linewidth}
Derivative of Hyperbolic Trigonometric Functions\\
\begin{tabular}{ll}
$(\sinh x)' = \cosh x$ & $(\csch x)' = -\csch x \cdot \coth x$\\
$(\cosh x)' = \sinh x$ & $(\sech x)' = -\sech x \cdot \tanh x$\\
$(\tanh x)' = \sech^{2} x$ & $(\coth x)' = -\csch^2 x$\\
\end{tabular}
\end{minipage}

\vspace{0.5cm}

\begin{minipage}{0.5\linewidth}
  Derivative of Inverse Trigonometric Functions\\
\begin{tabular}{ll}
$(\sin^{-1} x)' = \displaystyle\frac{1}{\sqrt{1-x^2}}$ & $(\csc^{-1} x )' = \displaystyle - \frac{1}{|x|\sqrt{x^2 - 1}}$\\
$(\cos^{-1} x)' = - \displaystyle\frac{1}{\sqrt{1-x^2}}$ & $(\sec^{-1} x )' = \displaystyle - \frac{1}{|x|\sqrt{x^2 - 1}}$\\
$(\tan^{-1} x )' = \displaystyle\frac{1}{1+x^2}$ & $(\cot^{-1} x )' = -\displaystyle\frac{1}{1 + x^2}$
\end{tabular}
\end{minipage}

\end{center}

\section{Uniqueness of Limits}

\begin{center}
{\renewcommand{\arraystretch}{1.3}
\begin{tabular}{l}
$\displaystyle \lim_{x \to c} f(x) = L \land \lim_{x \to c} f(x) = M \Longrightarrow L = M$\\
\textbf{Proof:}\\
Suppose the contrary that:\\
$\displaystyle \lim_{x \to c} f(x) = L \land \lim_{x \to c} f(x) = M, L \neq M$\\
Assume that $L > M, L = M + K$, WLOG\\
Let's choose $\epsilon = \displaystyle\frac{k}{2}$, this way the intervals do not overlap.\\
$\displaystyle \lim_{x \to c} f(x) = L: \exists \delta_1 > 0, \ni: 0 < |x - c| < \delta_1 \rightarrow |f(x) - L| < \epsilon$\\
$\displaystyle \lim_{x \to c} f(x) = M: \exists \delta_2 > 0, \ni: 0 < |x - c| < \delta_2 \rightarrow |f(x) - M| < \epsilon$\\
Let $\delta = min(\delta_1, \delta_2)$ so that for any $\delta > 0$, $f(x) \in (M - \epsilon, M + \epsilon)$ and $f(x) \in (L - \epsilon, L + \epsilon)$.\\
Contradiction: This is impossible since we set $\epsilon = \displaystyle\frac{k}{2}$ to guarantee intervals do not overlap.\\
Therefore, $\displaystyle \lim_{x \to c} f(x) = L \land \lim_{x \to c} f(x) = M \Longrightarrow L = M$ by contradiction.
\end{tabular}}
\end{center}
\marginpar{$QED$}


\end{document}
