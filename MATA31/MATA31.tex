  
\documentclass{article}
\usepackage{amsfonts}
\usepackage{tikz}
\usepackage{amsmath}
\usepackage{amssymb}
\usepackage{setspace}
\usepackage{mathtools}
\usepackage{fancyhdr}
\usepackage{fullpage}
\usepackage{graphicx}
\usepackage{caption}
\usepackage{lipsum}
\usepackage{venndiagram}
\usepackage{booktabs}
\DeclareMathOperator{\sech}{sech}
\DeclareMathOperator{\csch}{csch}
\DeclareMathOperator{\arcsec}{arcsec}
\DeclareMathOperator{\arccot}{arcCot}
\DeclareMathOperator{\arccsc}{arcCsc}
\DeclareMathOperator{\arccosh}{arcCosh}
\DeclareMathOperator{\arcsinh}{arcsinh}
\DeclareMathOperator{\arctanh}{arctanh}
\DeclareMathOperator{\arcsech}{arcsech}
\DeclareMathOperator{\arccsch}{arcCsch}
\DeclareMathOperator{\arccoth}{arcCoth}
\graphicspath{ {./images/} }

\begin{document}

\begin{titlepage}
\begin{center}
  \vspace{5cm}

  {\textbf{MATA31}}\\

  Calculus 1, for Mathematical Sciences, Fall 2018

  \hrulefill
  \vspace{0.3cm}

  \textbf{\huge Various Core Proofs/Identities}

  \hrulefill

  \begin{tabular}{ll}
  Current Instructor: & Natalia Breuss\\
  \end{tabular}

  \vspace{0.5cm}

  \includegraphics[scale = 0.25]{UTSC}

  \vspace{5cm}

  \end{center}

  \textbf{Instructor(s):}

  \begin{center}

  \begin{tabular}{ll}
   &\textbf{Dr. Natalia Breuss}\\
  \textbf{Email:} & n.breuss@utoronto.ca\\
  \textbf{Office:} & IC484\\
  \textbf{Office Hours:}& Wednesday 11:00 - 13:00\\
  \end{tabular}

  \end{center}

\end{titlepage}
\pagestyle{fancy}

\vspace{0.5cm}
\chead{Richard Hong}
\lhead{MATA31 - Core Proofs/Identities}
\rhead{\leftmark}
\cfoot{}
\rfoot{Page \thepage}

\text{}

\section{Identities}
\begin{center}

\textbf{Basic Trigonometric Identities}

\begin{minipage}{0.3\linewidth}
Reciprocal Identities\\
\begin{tabular}{l}
$(\sin x)^{-1} = \csc x$\\
$(\cos x)^{-1} = \sec x$\\
$(\tan x)^{-1} = \cot x$
\end{tabular}
\end{minipage}
\begin{minipage}{0.3\linewidth}
Pythagorean Identities\\
\begin{tabular}{l}
$\sin^2 x + \cos^x = 1$\\
$\tan^2 x + 1 = \sec^2 x$\\
$1 + \cot^2 x = \csc^2 x$
\end{tabular}
\end{minipage}
\begin{minipage}{0.3\linewidth}
Angle Addition Identities\\
\begin{tabular}{l}
$\sin(x \pm y) = \sin x \cdot \cos y \pm \cos x \cdot \sin y$\\
$\cos(x + y) = \cos x \cdot \cos y - \sin x \cdot \sin y$\\
$\cos(x - y) = \cos x \cdot \cos y + \sin x \cdot \sin y$\\
$\tan(x + y) = \displaystyle\frac{\tan x + \tan y}{1 - \tan x \cdot \tan y}$\\
$\tan(x - y) = \displaystyle\frac{\tan x - \tan y}{1 + \tan x \cdot \tan y}$
\end{tabular}
\end{minipage}

\vspace{0.5cm}

\textbf{Hyperbolic Trigonometric Identities}

\vspace{0.3cm}

{\renewcommand{\arraystretch}{2}
\begin{tabular}{ll}
$\sinh x = \dfrac{e^x - e^{-x}}{2}$ & $\csch x = \dfrac{2}{e^x - e^{-x}}$\\
 $\cosh x = \dfrac{e^x + e^{-x}}{2}$ & $\sech x = \dfrac{2}{e^x + e^{-x}}$\\
 $\tanh x = \dfrac{e^x - e^{-x}}{e^x + e^{-x}}$ & $\coth x = \dfrac{e^x + e^{-x}}{e^x - e^{-x}}$
\end{tabular}}

\vspace{0.5cm}

\textbf{Derivatives of Trigonometric Functions}

\vspace{0.3cm}

\begin{minipage}{0.4\linewidth}

\centerline{Derivative of Trigonometric Functions}

\begin{tabular}{ll}
$(\sin x)' = \cos x$ & $(\csc x)' = -\csc x \cot x$\\
$(\cos x)' = -\sin x$ & $(\sec x)' = \sec x \tan x$\\
$(\tan x)' = \sec^2 x$ & $(\cot x)' = -\csc^2 x$
\end{tabular}
\end{minipage}
\hspace{0.3cm}
\begin{minipage}{0.55\linewidth}

\centerline{Derivative of Hyperbolic Trigonometric Functions}

\begin{tabular}{ll}
$(\sinh x)' = \cosh x$ & $(\csch x)' = -\csch x \cdot \coth x$\\
$(\cosh x)' = \sinh x$ & $(\sech x)' = -\sech x \cdot \tanh x$\\
$(\tanh x)' = \sech^{2} x$ & $(\coth x)' = -\csch^2 x$\\
\end{tabular}
\end{minipage}

\vspace{0.5cm}

\begin{minipage}{0.45\linewidth}

\centerline{Derivative of Inverse Trigonometric Functions}

\begin{tabular}{ll}
$(\sin^{-1} x)' = \displaystyle\frac{1}{\sqrt{1-x^2}}$ & $(\csc^{-1} x )' = \displaystyle - \frac{1}{|x|\sqrt{x^2 - 1}}$\\
$(\cos^{-1} x)' = - \displaystyle\frac{1}{\sqrt{1-x^2}}$ & $(\sec^{-1} x )' = \displaystyle \frac{1}{|x|\sqrt{x^2 - 1}}$\\
$(\tan^{-1} x )' = \displaystyle\frac{1}{1+x^2}$ & $(\cot^{-1} x )' = -\displaystyle\frac{1}{1 + x^2}$
\end{tabular}
\end{minipage}

\pagebreak

\text{}

\vspace{0.7cm}


\textbf{Derivatives of common functions}

\vspace{0.3cm}

\begin{minipage}{0.4\textwidth}
  \begin{tabular}{l}
$\dfrac{d}{dx} \ln x = \dfrac{1}{x}$\\
\\
$\dfrac{d}{dx} e^x = e^x \cdot \ln(e)$\\
\\
$\dfrac{d}{dx} e^{2x} = e^x \cdot \ln(e) \cdot \dfrac{d}{dx} x = e^x \cdot 2x$
\end{tabular}
\end{minipage}
\begin{minipage}{0.5\textwidth}
\begin{tabular}{ll}
$\dfrac{d}{dx} x^n = nx^{(n-1)}$ & $\dfrac{d}{dx} f \cdot g = f'\cdot g + g' \cdot f$\\
\\
$\dfrac{d}{dx} f + h = f' + h'$ & $\dfrac{d}{dx} \dfrac{f}{g} = \dfrac{g\cdot f' - f \cdot g'}{g^2}$\\
\\
$\dfrac{d}{dx} f(g) = f'(g) \cdot g'$
\end{tabular}
\end{minipage}


\end{center}



\section{Uniqueness of Limits}

\textbf{Theorem.} $\displaystyle \lim_{x \to c} f(x) = L \land \lim_{x \to c} f(x) = M \Longrightarrow L = M$

\begin{center}
{\renewcommand{\arraystretch}{1.3}
\begin{tabular}{l}
$\displaystyle \lim_{x \to c} f(x) = L \land \lim_{x \to c} f(x) = M \Longrightarrow L = M$\\
\textbf{Proof:}\\
Suppose the contrary that:\\
$\displaystyle \lim_{x \to c} f(x) = L \land \lim_{x \to c} f(x) = M, L \neq M$\\
Assume that $L > M, L = M + K$, WLOG\\
Let's choose $\epsilon = \displaystyle\frac{k}{2}$, this way the intervals do not overlap.\\
$\displaystyle \lim_{x \to c} f(x) = L: \exists \delta_1 > 0, \ni: 0 < |x - c| < \delta_1 \rightarrow |f(x) - L| < \epsilon$\\
$\displaystyle \lim_{x \to c} f(x) = M: \exists \delta_2 > 0, \ni: 0 < |x - c| < \delta_2 \rightarrow |f(x) - M| < \epsilon$\\
Let $\delta = min(\delta_1, \delta_2)$ so that for any $\delta > 0$, $f(x) \in (M - \epsilon, M + \epsilon)$ and $f(x) \in (L - \epsilon, L + \epsilon)$.\\
Contradiction: This is impossible since we set $\epsilon = \displaystyle\frac{k}{2}$ to guarantee intervals do not overlap.\\
Therefore, $\displaystyle \lim_{x \to c} f(x) = L \land \lim_{x \to c} f(x) = M \Longrightarrow L = M$ by contradiction.
\end{tabular}}
\end{center}

\marginpar{$QED$}

\section{One Sided Limits}

\textbf{Theorem.} $\displaystyle \lim_{x \to c} f(x) = L \Longleftrightarrow \lim_{x \to c^-} f(x) = \lim_{x \to c^+} f(x) = L$

\vspace{0.3cm}

We need to prove that:

\textbf{1. } $\displaystyle \lim_{x \to c} f(x) = L \Rightarrow \lim_{x \to c^-} f(x) = \lim_{x \to c^+} f(x) = L$

\textbf{2. } $\displaystyle \lim_{x \to c^-} f(x) = \lim_{x \to c^+} f(x) = L \Rightarrow \lim_{X\to c} f(x) = L $

\begin{center}
  \begin{minipage}{0.5\linewidth}
  {\renewcommand{\arraystretch}{1.3}
\begin{tabular}{l}
\textbf{1. }Assume $\displaystyle \lim_{x \to c} f(x) = L$\\
\hline
$\forall \epsilon > 0, \exists \delta > 0, \ni: 0 < |x - c| < \delta \Rightarrow |f(x) - L| < \epsilon$\\
$0 < |x - c| < \delta \equiv x \in (c - \delta, c) \cup (c, c + \delta)$\\
$x \in (c - \delta, c) \Rightarrow |f(x) - L| < \epsilon$\\
$x \in (c, c + \delta) \Rightarrow |f(x) - L| < \epsilon$\\
$\therefore \displaystyle \lim_{x \to c^-} f(x) = \lim_{x \to c^+} f(x) = L$
\end{tabular}}

\end{minipage}
\hspace{0.5cm}
\begin{minipage}{0.4\linewidth}
    {\renewcommand{\arraystretch}{1.3}
  \begin{tabular}{l}
\textbf{2. }Assume $\displaystyle \lim_{x \to c^-} f(x) = \lim_{x \to c^+} f(x) = L$\\
\hline
$\forall \epsilon > 0, \exists \delta_1, \delta_2 > 0, \ni:$\\
$x \in (c - \delta, c) \lor x \in (c, c + \delta) \Rightarrow |f(x) - L| < \epsilon$\\
Let $\delta = \min(\delta_1,\delta_2)$\\
$0 < |x - c| < \delta \Rightarrow |f(x) - L| < \epsilon$\\
By definition: $\displaystyle \lim_{x \to c} f(x) = L$
\end{tabular}}

\end{minipage}

\end{center}

\marginpar{$QED$}


\pagebreak

\text{}

\section{Mean Value Theorem}

\begin{tabular}{l}
if function f is continuous across $[a,b]$ and differentiable across$(a,b)$\\
then, there exists atleast one $c \in (a,b)$ such that $f'(c) = \dfrac{f(b) - f(a)}{b - a}$\\
This means if the conditions satisfy, there will be atleast one point c where the instataneous\\
rate of change is the same as the average rate of change.\\
\end{tabular}


\pagebreak

\text{}

\section{Applications of MVT}

Suppose $f(x)$ is continuous on $[1,5]$ and differentiable on (1,5) and $f'(x) < \dfrac{3}{8}, \forall x \in (1,5)$. If $f(1) = 1$, show that $f(x) < \dfrac{5}{2}, \forall x \in [1.5]$

Since f is continuous and differentiable...by Mean Value Theorem, $\exists c \in (1,5), \ni: f'(c) = \dfrac{f(5) - f(1)}{5-1}$

let $x \in (1,5), f'(x) = \dfrac{f(x) - f(1)}{x-1} < \dfrac{3}{8} \Longleftrightarrow \dfrac{f(x) - 1}{x - 1} < \dfrac{3}{8}$

$f(x) - 1 < \dfrac{3}{8}(x - 1)$, $f(x) < \dfrac{3}{8} (x-1) + 1$

$f(x) < \dfrac{3}{8}(5 - 1) + 1 \rightarrow f(x) < \dfrac{5}{2}$ \marginpar{$QED$}

\vspace{1cm}

Suppose $f(x)$ is odd for all x and diff across every real number. Prove that for every positive number b, there exists a positive number c, in $(-b,b)$ such that $f'(c) = \dfrac{f(b)}{b}$

Given $b > 0, f(-x) = - f(x)$

since $f$ is differentiable on $(-\infty,\infty) \Longrightarrow f $ is continuous on $(-\infty,\infty)$

$f$ is cont on [-b,b] and differentiable on (-b,b)

becuase (-b,b)$\subset(-\infty, \infty)$,

By MVT, $\exists c \in (-b , b) \ni: f'(c) = \dfrac{f(b) - f(-b)}{b - (-b)} = \dfrac{f(b) + f(b)}{b + b} = \dfrac{f(b)}{b}$

\section{Rolle's Theorem}

Suppose f is continuous on [a,b] and differentiable on (a,b) and $f(a) = f(b)$ then there exists atleast a $c \in (a,b)$ such that $f'(c) = 0$

\section{Applications R'T}

Show that the function $2x + \cos x$ has exactly one real root.

Let $f(x) = 2x + \cos x$

$f(- \pi) = -2\pi + \cos(-\pi) = -2pi - 1 < 0$

$f(0) = 2 + \cos(0) = 1 > 0$

Since $f(x)$ is a sum of a polynomial and periodic trigonometric function, f is continuous and differentiable for all x, By IVT, $\exists c \in (-\pi ,0) \ni: f(c) = 0$

Suppose $f(x)$ ias two roots  on a,b a < b, then $f(a) = f(b) = 0$, Since f is continuous on [a,b] and differentiable on open interval (a,b).

By Rolle's Theorem $\exists r \in (a,b) \ni: f'(r) = 0$

$f'(x) = 2 - \sin(x) > 0$, Contradiction, rolles theorem fails and therefore there must be exactly one root and one root only. \marginpar{$QED$}

\pagebreak

\text{}

\section{Fermat's Theorem}


If $f$ has a local min/max at x = c and $f'(c)$ exists, then f'(c) = 0.

$f'(x) = 0$ or undefined for c-pts:

$f'(x) > 0 \Rightarrow f(x)$ is increasing

$f'(x) < 0 \Rightarrow f(x)$ is decreasing

\section{Proof of Sum Law for Limits}

Prove $\displaystyle \lim_{x \to c} f(x) = L \land \lim_{x \to c} g(x) = M \Longrightarrow \lim_{x \to c} [f(x) \pm g(x)] = L \pm M$

Suppose $\displaystyle \lim_{x \to c} f(x) = L \land \lim_{x \to c} g(x) = M$

$\displaystyle \lim_{x \to c} f(x) = L$

$\forall \epsilon_1 > 0, \exists \delta_1 > 0, \ni : 0 < |x - c| <\delta_1 \Longrightarrow |f(x) - L| < \epsilon_1$

\vspace{0.3cm}

$\displaystyle \lim_{x \to c} g(x) = M$

$\forall \epsilon_2 > 0, \exists \delta_2 > 0, \ni : 0 < |x - c| <\delta_1 \Longrightarrow |g(x) - M| < \epsilon_2$

Consider $\epsilon_1, \epsilon_2 = \dfrac{\epsilon}{2}$

There must consequently be a $\delta_1 > 0$ and a $\delta_2 > 0$ such that:

$0 < |x - c| < \delta_1 \Longrightarrow |f(x) - L| < \dfrac{\epsilon}{2}$

$0 < |x - c| < \delta_1 \Longrightarrow |g(x) - M| < \dfrac{\epsilon}{2}$

let $\delta = \min(\delta_1, \delta_2) $ such that $ 0 <|x - c| < \delta \Longrightarrow |f(x) - L| < \dfrac{\epsilon}{2} \land |g(x) - M| < \dfrac{\epsilon}{2}$

$0 <|x - c| < \delta \Longrightarrow |f(x) - L| + |g(x) - M| < \epsilon$

$0 <|x - c| < \delta \Longrightarrow |f(x) - L + (g(x) - M)| \leq |f(x) - L| + |g(x) - M| < \epsilon$

$0 <|x - c| < \delta \Longrightarrow |f(x) + g(x) - (L + M)| < \epsilon$

\vspace{0.3cm}

Thus: $\displaystyle \lim_{x\to c} f(x) \pm g(x) = L \pm M$

and: $\displaystyle \lim_{x \to c} f(x) = L \land \lim_{x \to c} g(x) = M \Longrightarrow \lim_{x \to c} [f(x) \pm g(x)] = L \pm M$ \marginpar{$QED$}

\section{Proof of Limits and Derivatives}

\begin{minipage}{0.5\linewidth}
Prove $\displaystyle \lim_{x\to 0} \dfrac{1}{x^2} = \infty$

$\forall M > 0, \exists \delta > 0, \ni : 0 < |x | < \delta \Longrightarrow \dfrac{1}{x^2} > M$

$\dfrac{1}{x^2} > M \rightarrow x^2 < \dfrac{1}{M} \rightarrow x < \sqrt{\dfrac{1}{M}}$

Proof:

Let $\delta = \sqrt{1}{M}$

$|x| < \delta \rightarrow \dfrac{1}{|x|} > {M} \rightarrow \dfrac{1}{|x|^2} > M \rightarrow \dfrac{1}{x^2} > M$
\end{minipage}
\begin{minipage}{0.5\linewidth}
Prove $\dfrac{d}{dx}[\ln x] = \dfrac{1}{x}$

$\dfrac{d}{dx}[\ln x] = \displaystyle \lim_{h \to 0} \dfrac{\ln(x + h) - \ln(x)}{h} = \lim_{h \to 0} \dfrac{\ln(\dfrac{x + h}{x})}{h}$

\vspace{0.2cm}

$\displaystyle = \lim_{h \to 0} \dfrac{1 + \dfrac{h}{x}}{h} = \lim_{h \to 0} (\dfrac{1}{h})[\ln(1 + \dfrac{h}{x})] = \lim_{h \to 0} [\ln(1 + \dfrac{h}{x})^{\dfrac{1}{h}}]$

\vspace{0.2cm}

let $n = \dfrac{h}{x}; h = nx; \dfrac{1}{h} = \dfrac{1}{nx}$

\vspace{0.2cm}

$\displaystyle \lim_{x\to 0} [(\ln(1 + n)^{\dfrac{1}{n}})^{\dfrac{1}{x}}] = \lim_{x \to 0} \dfrac{1}{x} \cdot [\ln(1 + n)^{\dfrac{1}{n}}] = \dfrac{1}{x} \cdot \ln e$

$ = \dfrac{1}{x}$

\end{minipage}
\marginpar{$QED$}

\pagebreak

\text{}




\end{document}